\section{Conclusion}
In this data exploration, we find gene candidates that could potentially run the cell cycle in C. Neoformans. Additionally, spectral clustering gives us bins on which genes in C. Neoformans are similar to one another, however, it is difficult to assess the quality of these bins. We discussed the results of applying spectral clustering with 24 clusters, using a $k$-nearest neighbors graph of 10 neighbors, where the edges were weighted by similarity. By visual inspection, these set of parameters gave better results than vastly increasing the number of neighbors or using the weighted complete graph. However, we are not sure what the best set of parameters would be for applying spectral clustering. This is something that could be explored further. Lastly, we used the backbone distance to perform this analysis due to its features of conserving order, being robust to noise, and capturing shape. However, it is possible that we may need to consider timing of extrema, which is something we are currently neglecting with the backbone distance. If we find that the timing of the peaks is important for our question of study, then we may need to redo the analysis using a time sensitive distance such as an $L_p$-norm. The next step is to meet with our biology collaborators at Duke University in order to determine which mathematical analysis methods would be most meaningful for studying these datasets of yeast organisms.